\documentclass{article}

\usepackage{lmodern}
\usepackage[T1]{fontenc}
\usepackage[spanish,activeacute]{babel}
\usepackage{mathtools}
\usepackage{enumerate} 

\title{Toma de actas}
\author{Daniel David Abell\'an Serrano}

\begin{document}
\maketitle
\newpage
\section*{'Indice}
	\subsection{Introducci'on}
		\begin{enumerate}
			\item Resumen
			\item Motivaci'on y objetivos
		\end{enumerate}
	\subsection{Estado del arte}
		\begin{enumerate}
			\item Plataformas
			\item Herramientas
			\item Tecnolog'ias
		\end{enumerate}
	\subsection{Metodolig'ia y tecnolog'ia}
		\begin{enumerate}
			\item Metodolog'ia
			\item Tecnolog'ias
		\end{enumerate}	
	\subsection{Desarrollo e implementaci'on}
		\begin{enumerate}
			\item Teem
		\end{enumerate}	
	\subsection{Producto final}
		\begin{enumerate}[a)]
			\item Prototipo funcional
			\item Pruebas
		\end{enumerate}	
	\subsection{Conclusiones y futuro}
	\subsection{Bibliograf'ia}
	\subsection{Ap'endices}
\newpage
\section{Introducci'on}
Resumen (Tambi'en en ingl'es).
(No hay resumen hasta que acabemos)

Motivaciones y objetivos.
Debido a la gran velocidad de avance tecnol'ogico en las TIC's es necesario adaptar y acercar estas tecnolog'ias a la sociedad para su uso cotidiano. En particular hemos creado una herramienta que facilita la organizaci'on y dinamismo de entidades colaborativas ayudando al registro y difusi'on de sus actividades a trav'es de internet en la plataforma Teem. En concreto hemos modificado Teem añadi'endole la funcionalidad necesaria para que se convierta en una herramienta de toma de actas en tiempo real as'i como mantener un registro público y duradero de cada entrada.

\section{Estado del arte}
Actualmente ya existen bastantes tecnologías que permiten la interacción entre personas y distribución de información en tiempo real. En particular nos centraremos en aquellas tecnologías y herramientas dedicadas a apoyar reuniones de trabajo en diferentes contextos y haremos más énfasis en aquellas que sean software libre.
	\subsection{Plataformas y herramientas}
		\begin{itemize}
			\item MeetingBunner, WebEx son plataformas que ofrecen la posibilidad de llevar a cabo reuniones de trabajo de manera remota a un grupo de personas considerable. Estos servicios tienen versiones de prueba sin coste alguno pero con funcionalidad limitada. Son sus servicios de pago los que dan una mayor cobertura a las necesidades que precise cada grupo de trabajo.
			En estas plataformas podemos encontrar herramientas muy diversas, entre ellas compartición de pantalla, edición en tiempo real de documentos, videoconferencia, compartición de archivos, chat o la posibilidad de grabar las reuniones.
			\item Meeting.gs, la gran característica de esta plataforma es la integración con distintas aplicaciones y plataformas como skype, Google Apps o Microsoft Office entre otras.
			\item Twiddla es una pizarra interactiva y en tiempo real que permite crear todo tipo de diagramas, gráficos y dibujos.
			\item Google Docs. Este conjunto de aplicaciones ofrece un pack de aplicaciones de ofimática integradas en la web que permiten la edición de cada archivo en tiempo real así como su difusión entre usuarios de google a través de invitaciones o externos mediante links especiales. Este conjunto de herramientas incluye toda la potencia que la red social de google ofrece al estar integradas en su servicio de almacenamiento Google Drive.
		\end{itemize}
	\subsection{Tecnolog'ias}
	Actualmente existen infinidad de herramientas y tecnologías para el desarrollo web; solo presentaremos las más importantes y relevantes para este proyecto.
	(Esta parte prefiero que la discutamos entre todos/as ya que sabéis más perooo habra que hablar de ángular, swellrt, etc)
\section{Metodolog'ia y tecnolog'ia}
	\subsection{Metodolog'ias}
	Dado que nuestro proyecto consiste en añadir funcionalidad a una herramienta ya existente hemos dividido el proceso en dos bloques principales:
		\begin{itemize}
			\item Proceso general. Durante todo el proceso de desarrollo del proyecto hemos seguido una rutina de reuniones semanales entre los integrantes del grupo y quincenales entre el grupo y nuestros tutores. Al empezar la fase de implementación las reuniones con los tutores pasaron a ser semanales para ir revisando el código y posibles dudas al respecto. (habrá que completarlo)
			\item Investigación y prototipado. En esta fase nos hemos dedicado en paralelo a estudiar las tecnologías necesarias para la posterior implementación del proyecto y a la fase de investigación previa. El estudio de las tecnologías las ha llevado cada miembro del grupo por su parte.
			En la fase de investigación nos hemos basado (y no seguido al pie de la letra) el diseño centrado en el usuario.
			\begin{itemize}
				\item En primer lugar tuvimos una toma de contacto donde expresamos qué nos sugería el proyecto así como un pequeño brainstorming donde propusimos soluciones rápidas al proyecto. (aquí tendremos que insertar esas ideas/sugerencias). Además también acordamos el ámbito al que está orientado nuestro proyecto, en concreto a asociaciones del procomún y entidades colaborativas.
				\item Después hicimos una pequeña investigación sobre el estado del arte la cual se puede consultar en el apartado correspondiente.
				\item Preparamos una serie de entrevistas a potenciales usuarios/as para crear el perfil ideal de usuario/a de nuestra aplicación. Estas reuniones consistieron en una entrevista con una batería de preguntas (incluimos la batería y un link a los audios?).
				(Aquí nuestra conclusión, lista de factoides y el perfil de Francisco)
				\item También preparamos un cuestionario el cual difundimos pero no tuvo mucho éxito así que no ha tenido influencia alguna en nuestras conclusiones.
				\item En último lugar y en base a los factoides y el usuario creado identificamos los problemas a resolver que ordenamos en una lista por prioridad de desarrollo. (Aquí deberíamos contar cómo y por qué los hemos ordenado así)
			\end{itemize}
		\item Implementación y pruebas (este para el final). 
		\end{itemize}
	\subsection{Tecnolog'ias}
\section{Desarrollo e implementaci'on}
\section{Producto final}
\section{Conclusiones y futuro}
\section{Bibliograf'ia}
\section{Ap'endices}

\end{document}